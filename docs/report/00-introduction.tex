\phantomsection\section*{ВВЕДЕНИЕ}\addcontentsline{toc}{section}{ВВЕДЕНИЕ}

В современном мире немаловажное значение имеет корпусная лингвистика.
Корпуса текстов находят применение в различных областях --- в машинном переводе, в разработке словарей, в лингвистических исследованиях.
Для того, чтобы из корпуса текстов можно было извлекать пользу, тексты в нем должны быть размечены.
Существуют алгоритмы, позволяющие автоматически производить разметку, но для проверки ее корректности все равно требуется вмешательство человека.

На данный момент не существует открытых параллельных корпусов технических текстов.
Также нет открытых информационных систем, позволяющих одновременно
\begin{itemize}
    \item производить разметку текста в параллельном корпусе,
    \item производить поиск по параллельному корпусу,
    \item организовать удобную работу множества разметчиков.
\end{itemize}

Создание такой информационной системы позволит во многом автоматизировать рабочее место разметчиков параллельного корпуса.

Целью данной работы является разработка базы данных для автоматизации рабочего места разметчиков параллельного корпуса технических текстов.

Задачи курсового проекта:
\begin{itemize}
    \item провести анализ предметной области параллельных корпусов текстов;
    \item спроектировать сущности базы данных и ограничения целостности АРМ разметчика корпуса технических текстов;
    \item выбрать средства реализации базы данных и приложения;
    \item разработать сущности базы данных и реализовать ограничения целостности базы данных;
    \item описать интерфейс доступа к базе данных;
    \item исследовать зависимость времени ответа от количества запросов в секунду.
\end{itemize}
