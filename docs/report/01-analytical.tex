\section{Аналитическая часть}

В данной части будет идти речь о корпусах текстов, видах текстов и текстовых разметок. % XXX будет идти или идет?

\subsection{Корпуса текстов}

Корпусная лингвистика --- раздел компьютерной лингвистики, занимающийся разработкой общих принципов построения и использования лингвистических корпусов (корпусов текстов) с применением компьютерных технологий.
Под лингвистическим, или языковым, корпусом текстов понимается большой, представленный в машиночитаемом формате, унифицированный, структурированный, размеченный, филологически компетентный массив языковых данных, предназначенный для решения конкретных лингвистических задач. \cite[с. 5]{cl}

\subsubsection{Виды}

В таблице \ref{tab:coc} ниже представлена классификация корпусов по некоторым признакам.

\begin{table}[H]
\centering
        \caption{Классификация корпусов \cite[с. 16]{cl}}
		\label{tab:coc}
        \begin{adjustbox}{width=\textwidth}
            \begin{tabular}{|p{0.3\textwidth}|p{0.7\textwidth}|}
        \hline
        Признак & Тип корпуса \\
        \hline
        \hline
        Цель & многоцелевые, специализированные \\
        \hline
        % Параллельность & одноязычные, двуязычные, многоязычные \\
        Параллельность & параллельные, сопоставимые (псевдопараллельные) \\
        \hline
        Тип языковых данных & письменные, устные (речевые), смешанные \\
        \hline
        <<Литературность>> & литературные, диалектные, разговорные, терминологические, смешанные \\
        \hline
        Жанр & литературные, фольклорные, драматургические, публицистические \\
        \hline
        Назначение & исследовательские, иллюстративные \\
        \hline
        % Динамичность & динамические, статические \\
        % \hline
        Разметка & размеченные, неразмеченные \\
        \hline
        Характер разметки & морфологические, синтаксические, семантические, анафорические и т. д. \\
        \hline
        % Доступность & свободно доступные, коммерческие, закрытые \\
        % \hline
        Объем текстов & полнотекстовые, <<фрагментнотекстовые>> \\
        \hline
		\end{tabular}
        \end{adjustbox}
\end{table}

\newpage

В данной работе особое внимание уделяется параллельным корпусам, так как именно для работы с таким видом корпусов будет разрабатываться база данных. % XXX meh, уделяется; будет разрабатываться

% Параллельный корпус --- это двуязычный корпус, то есть текст оригинала и его перевод на какой-то другой язык, причем эти два текста не просто лежат рядом друг с другом, а должны быть выровнены: отдельные фрагменты оригинала должны совпадать с соответствующими фрагментами перевода. Именно это позволяет использовать параллельный корпус как инструмент исследования. 

Параллельный корпус --- это двуязычный корпус.
В нем хранится два множества текстов --- оригиналов и их переводов.
Работа с корпусом (выравнивание, разметка, поиск) производится сразу с двумя текстами. %, ровно один из которых оригинальный.

Для возможности использования параллельного корпуса в качестве инструмента исследования, тексты должны быть выровнены --- отдельные фрагменты оригинала должны совпадать с соответствующими фрагментами перевода \cite{postnauka}.

Подробнее о применении, устройстве и проблематике параллельных корпусов будет говориться в следующих подразделах.

\subsubsection{Применение}

В данном подразделе рассматриваются некоторые применения параллельных корпусов.

% 1) Машинный перевод
% Параллельный корпус является важнейшей частью работы статистических систем машинного перевода, а также систем, работа которых основана на правилах.
% Данные системы постоянно анализируют доступные тексты для вычисления наиболее вероятного перевода.
% Кроме того, параллельный корпус используется и в новейшем методе машинного перевода с использованием глубокого обучения.

% 2) Инструменты обработки естественного языка Некоторые инструменты NLP (аббр. Natural Language Processing) также зависят от параллельного
% корпуса. Примерами таких инструментов являются программы для кросс-языкового извлечения данных (Cross Language Information Retrieval), транслитерации, систем ответы на вопросы (Question Answering Systems) и др.

% 3) Языкознание Одно слово в языке может иметь множество значений в зависимости от контекста. Таким образом, параллельный корпус помогает в исследованиях влияния окружения на перевод слова на другой язык (при наличии достаточного количества загруженных текстов). С помощью корпуса выявляются характерные для языка паттерны использования конкретных слов, фраз. Кроме того, на основе текстов делаются выводы об особенностях морфологии, синтаксиса языков, а также вытекающих из них особенностях перевода. Самым ярким примером науки, которой необходим параллельный корпус, является контрастивная лингвистика.

% 4) Работа переводчика или преподавателя В связи с множеством возможных переводов слова или фразы в зависимости от контекста, работа переводчика неразрывно связана с использованием лингвистических инструментов. Как правило, 50\% времени перевода - это использование справочных материалов. Использование параллельного корпуса призвано облегчить работу переводчика, поскольку корпус содержит множество примеров перевода слов, фраз и предложений в зависимости от контекста их употребление. Кроме того, корпус может помочь людям, изучающим иностранный язык.

\subsubsection*{Машинный перевод}

\blindtext

\subsubsection*{Обработка естественного языка (NLP)}

\blindtext

\subsubsection*{Кросс-языковой поиск информации (CLIR)}

\blindtext

\subsubsection*{Разработка словарей и учебных материалов}

\blindtext

\subsubsection*{Лингвистические исследования}

\blindtext

% 1. Машинный перевод
% 2. Сравнительный анализ языков
% 3. Автоматическая обработка естественного языка (NLP)
% 4. Кросс-языковой поиск информации (CLIR)
% 5. Выявление и сопоставление терминов и специализированной лексики в различных языках.
% 6. Лингвистические исследования
% Изучение языкового варьирования и изменчивости: Анализ различий в использовании языка в разных культурных и социальных контекстах.
% Контрастивная лингвистика: Сравнение языков на разных уровнях (фонетическом, морфологическом, синтаксическом, лексическом) для изучения языковых особенностей и взаимосвязей.
% 5. Разработка лингвистических ресурсов
% Создание двуязычных и многоязычных словарей: Использование параллельных текстов для составления словарных статей с примерами употребления слов и выражений в различных языках.
% Разработка учебных материалов: Подготовка образовательных ресурсов для изучения иностранных языков, включая упражнения на перевод, сопоставление текстов и понимание лексических и грамматических структур.

\subsubsection{Устройство}

Параллельные корпуса устроены таким образом, чтобы решались поставленные перед ними задачи.
Типичная задача параллельных корпусов включает в себя поиск по корпусу по словам или по тегам.

Для осуществления поиска по корпусу, в нем должна храниться следующая информация:
\begin{itemize}
    \item тексты и их метаданные,
    \item выравнивание,
    \item теги и аннотации.
\end{itemize}

Теги и аннотации, наряду с выровненными текстами обычно [много ссылок] хранятся в формате XML.
Но хранение информации в таком формате имеет ряд недостатков:
\begin{itemize}
    \item избыточность (хмл теги, дублирование информации)
    \item долгий поиск (парсинг текстового файла, да ещё и хмл - с кучей лишней инфы в виде тегов)
    \item большой размер (можно хранить ту же самую информацию в виде бинарного файла и экономить место)
    \item TODO FIXME
\end{itemize}

\subsection{Тексты}

В данном разделе будут рассмотрены основные виды текстовых разметок, структура технических текстов, а также будет описана проблема, возникающая при автоматическом выравнивании текстов на уровне терминов.

\subsubsection{Виды разметок}

\subsubsection*{Морфологическая}

\blindtext

\subsubsection*{Синтаксическая}

\blindtext

\subsubsection*{Семантическая}

\blindtext

\subsubsection*{Метаразметка}

\blindtext

\subsubsection{Структура технических текстов}

\subsubsection{Проблема терминов}

Выравнивание текстов на уровне секций, абзацев и предложений обычно не представляет трудности, и часто такое выравнивание можно автоматизировать.
Проблемы возникают при попытке выровнять тексты на уровне терминов.
Автоматически такое выравнивание произвести бывает сложно.
Причина сложности автоматического выравнивания на уровне терминов будет рассмотрена ниже на примере фразеологизмов.

Набор слов в фразеологизмах может иметь разные значения в зависимости от контекста.
Например, предложение <<It was a piece of cake>> нельзя перевести однозначно, не зная контекста, в котором оно употреблено.

В контексте 
\begin{itemize}
    \item Was it difficult?
    \item It was a piece of cake.
\end{itemize}
его можно перевести, как <<было просто>>, а в контексте
\begin{itemize}
    \item What was in the box?
    \item It was a piece of cake.
\end{itemize}
оно точно имеет отношение к куску пирога.

Таким образом, для корректности перевода, машинный перевод должен учитывать контекст, в котором термин употреблен.
Но это и есть одна из задач, для решения которой параллельные корпуса и создаются изначально.
В этом и заключается проблема терминов.

% Терминология русского языка складывалась и продолжает складываться на основе англоязычной терминологии.
% Иногда получается, что давно существующий в русском языке термин приобретает новое значение, заимствованное из английского.
% Так, например, слово <<корпус>> приобрело новое значение от английского <<corpus>> --- {сборник текстов} \cite{cl}.
% В связи с этим, при разметке текстов, возникают трудности при сопоставлении терминов из одного языка с терминами другого.

% Также, в связи с языковыми и культурными различиями, случается, что какой-то термин из одного языка не имеет точного эквивалента в другом.
% Например, не имеет точного эквивалента в русском языке японский термин \begin{CJK*}{UTF8}{min}
% 侘寂
% \end{CJK*}~(Wabi-Sabi), который означает {умение восхищаться чем-то несовершенным}.
% Или наоборот, русский <<кипяток>> нельзя перевести на английский одним словом.

% \subsection{Разметки}

\subsection{Существующие параллельные корпуса}

% XML - неэффективно

\subsection{Вывод}
