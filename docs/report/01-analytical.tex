\section{Аналитический раздел}

\subsection{Анализ предметной области}

Корпусная лингвистика --- раздел компьютерной лингвистики, занимающийся разработкой общих принципов построения и использования лингвистических корпусов (корпусов текстов) с применением компьютерных технологий.~\cite[с. 11]{cl2022}

\subsubsection{Корпуса текстов}

Под лингвистическим, или языковым, корпусом текстов понимается большой, представленный в машиночитаемом формате, унифицированный, структурированный, размеченный, филологически компетентный массив языковых данных, предназначенный для решения конкретных лингвистических задач.~\cite[с. 11]{cl2022}

В понятие <<корпус текстов>> входит также поисковая система, позволяющая производить поиск по корпусу, называемая \textit{корпусным менеджером}.
Поиск в корпусе позволяет по любому слову построить \textit{конкорданс} --- список всех употреблений данного слова к контексте со ссылками на источник.~\cite[с. 12]{cl2022}

Выделяют как минимум три типа корпусов текстов:
\begin{itemize}
    \item корпусы первого типа — универсальные, отражающие все многообразие речевой деятельности;
    \item корпусы второго типа — специфичные, отражающие бытование некоторого языкового или культурного явления в общественной речевой практике, например корпус пословиц или корпус политических метафор в газетной речи;
    \item корпусы третьего типа — специфичные, создаваемые для решения специальной задачи, например для обучения, для задач социолингвистики, для отладки систем машинного перевода.~\cite[с. 12]{cl2022}
\end{itemize}

\subsubsection*{Виды корпусов текстов}

В таблице \ref{tab:cc} приведена классификация корпусов текстов по разным признакам.

\begin{table}[H]
    \centering
    \begin{tabular}{|p{5cm}|p{11cm}|}
        \hline
        \textbf{Признак} & \textbf{Типы корпусов} \\ \hline
        Цель & Многоцелевые, специализированные \\ \hline
        Параллельность & Параллельные, сопоставимые \\ \hline
        Тип языковых данных & Письменные, устные (речевые), смешанные \\ \hline
        «Литературность» & Литературные, диалектные, разговорные, терминологические, смешанные \\ \hline
        Жанр & Литературные, фольклорные, драматургические, публицистические \\ \hline
        Назначение & Исследовательские, иллюстративные \\ \hline
        Динамичность & Динамические (мониторные), статические \\ \hline
        Разметка & Размеченные, неразмеченные \\ \hline
        Характер разметки & Морфологические, синтаксические, семантические, анафорические, просодические и т. д. \\ \hline
        Доступность & Свободно доступные, коммерческие, закрытые \\ \hline
        Объем текстов & Полнотекстовые, «фрагментнотекстовые» \\ \hline
    \end{tabular}
    \caption{Классификация корпусов~\cite[с. 57]{cl2022}}
    \label{tab:cc}
\end{table}

Корпус технических текстов, для которого будет разрабатываться база данных в настоящей работе, относится к корпусам третьего типа и является:
\begin{itemize}
    \item специализированным,
    \item параллельным,
    \item многоязыковым,
    \item письменным,
    \item терминологическим,
    \item динамическим (постоянно будет пополняться),
    \item размеченным,
    \item свободно доступным,
    \item полнотекстовым.
\end{itemize}

\subsubsection*{Применение корпусов текстов}

\subsubsection{Тексты и разметки}

\subsubsection*{Классификация текстов}

\subsubsection*{Структура технических текстов}

\subsubsection*{Виды текстовых разметок}

\subsection{Существующие решения}

\subsection{Формализация задачи}

\subsection{Формализация данных}

\subsection{Сущности базы данных}

% ERD в нотации Чена

\subsection{Формализация и описание пользователей}

\subsection{Сценарии использования}

% Use-Case диаграмма

\subsection{Анализ существующих баз данных}

\subsubsection{Выбор базы данных}

\subsection{Вывод}
