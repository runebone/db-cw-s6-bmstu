\phantomsection\section*{РЕФЕРАТ}\addcontentsline{toc}{section}{РЕФЕРАТ}

Отчет \pageref{LastPage} с., \total{figure} рис., \total{table} табл., 29 источн., 2 прил.

КОРПУСНАЯ ЛИНГВИСТИКА,
ПАРАЛЛЕЛЬНЫЕ КОРПУСА ТЕКСТОВ,
ТЕРМИНОЛОГИЧЕСКАЯ РАЗМЕТКА,
БАЗА ДАННЫХ,
РЕЛЯЦИОННАЯ МОДЕЛЬ,
POSTGRESQL,
KEYDB,
LOCUST

Цель работы --- разработка базы данных для автоматизации рабочего места разметчиков параллельного корпуса технических текстов.

В данной работе был проведен анализ предметной области корпусов текстов, рассмотрены существующие модели данных и была выбрана модель, на основе которой проводилась дальнейшая разработка базы данных.
Были спроектированы и реализованы 10 сущностей базы данных, ограничения целостности, ролевая модель на уровне базы данных.
Было описан интерфейс для доступа к базе данных и было разработано Web-приложение для взаимодействия с базой данных.
Было проведено исследование зависимости времени ответа от количества запросов в секунду и проведен сравнительный анализ эффективности реализаций приложения с использованием дополнительного кеширования и без него.
В результате исследования был получен следующий результат --- при 752 запросах в секунду использование кеша позволяет ускорить среднее время ответа более, чем в 3 раза.
