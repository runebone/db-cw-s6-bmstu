\phantomsection\section*{РЕФЕРАТ}\addcontentsline{toc}{section}{РЕФЕРАТ}

Отчет \pageref{LastPage} с., \total{figure} рис., \total{table} табл., 29 источн., 2 прил.

КОРПУСНАЯ ЛИНГВИСТИКА,
ПАРАЛЛЕЛЬНЫЕ КОРПУСА ТЕКСТОВ,
ТЕРМИНОЛОГИЧЕСКАЯ РАЗМЕТКА,
БАЗА ДАННЫХ,
РЕЛЯЦИОННАЯ МОДЕЛЬ,
POSTGRESQL,
LOCUST

% Пример из ГОСТа
% Объектом исследования являются поршневые установки для точного воспроизведения и измерения больших расходов газа.
% Цель работы — разработка методики метрологических исследований установок и нестандартной аппаратуры для их осуществления.
% В процессе работы проводились экспериментальные исследования отдельных составляющих и общей погрешности установок.
% В результате исследования впервые были созданы две поршневые реверсивные расходомерные установки: первая на расходы до 0,07 м3/с, вторая — до 0,33 м3/с.
% Основные конструктивные и технико-эксплуатационные показатели: высокая точность измерения при больших значениях расхода газа.
% Степень внедрения — вторая установка по разработанной методике аттестована как образцовая.
% Эффективность установок определяется их малым влиянием на ход измеряемых процессов. Обе установки могут применяться для градуировки и поверки промышленных ротационных счетчиков газа, а также тахометрических расходомеров.
