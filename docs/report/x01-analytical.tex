\section{Аналитическая часть}

В данной части вводятся понятия корпусов текстов и текстовых разметок.
Анализируются основные виды текстовых корпусов, текстовых разметок и их структура.
Описывается структура технических текстов, и рассматривается связанная с ними проблематика.

\subsection{Корпуса текстов}

% По одному из определений, под корпусом текстов понимается большой унифицированный, структурированный, размеченный, филологически компетентный массив языковых данных, предназначенный для решения конкретных лингвистических задач \cite{kl}.

% В данном разделе дается определение корпусу текстов, TODO

По Т. МакЭнери и Э. Вилсону, под корпусом текстов понимается {собрание языковых фрагментов, отобранных в соответствии с четкими языковыми критериями для использования в качестве модели языка} \cite{kl}.

% Зачем нужны
\subsubsection{Применения}

Корпуса текстов используются для различных лингвистических, образовательных и технологических целей, например, для выявления закономерностей в языке, анализа структуры технических текстов, обработки естесственного языка, составления словарей или интерактивного обучения языку.

% Корпуса текстов используются в различных сферах человеческой деятельности:
% \begin{itemize}
%     \item Лингвистика и языкознание;
%     \item Машинный перевод;
%     \item Разработка систем обработки естественного языка (англ. NLP, Natural Language Processing);
%     \item Обучение языкам;
%     \item Составление словарей;
%     \item Социолингвистика.
% \end{itemize}

% Какими бывают
\subsubsection{Типы и классификация}

% XXX Do I really need that information?
Различают несколько типов текстовых корпусов по степени <<универсальности>> \cite{kl}:
\begin{enumerate}
    \item корпусы первого типа --- универсальные, отражают в себе все многообразие речевой деятельности;
    \item корпусы второго типа --- специфичные, создаваемые для специальной цели, отражают некоторое языковое или культурное явление в общественной речевой практике, например, корпус пословиц или анекдотов;
    \item корпусы третьего типа --- специфичные, создаваемые для решения специальной задачи, например, для отладки систем машинного перевода.
\end{enumerate}

Текстовые корпуса различают и по другим признакам.
Некоторые из них представлены ниже, в таблице \ref{tab:coc}.
\begin{table}[H]
\centering
        \caption{Классификация корпусов \cite[с. 16]{kl}}
		\label{tab:coc}
        \begin{adjustbox}{width=1\textwidth}
            \begin{tabular}{|p{0.3\textwidth}|p{0.7\textwidth}|}
        \hline
        Признак & Тип корпуса \\
        \hline
        \hline
        Цель & многоцелевые, специализированные \\
        \hline
        Параллельность & одноязычные, двуязычные, многоязычные \\
        \hline
        Тип языковых данных & письменные, устные (речевые), смешанные \\
        \hline
        <<Литературность>> & литературные, диалектные, разговорные, терминологические, смешанные \\
        \hline
        Жанр & литературные, фольклорные, драматургические, публицистические \\
        \hline
        Назначение & исследовательские, иллюстративные \\
        \hline
        Динамичность & динамические, статические \\
        \hline
        Разметка & размеченные, неразмеченные \\
        \hline
        Характер разметки & морфологические, синтаксические, семантические, анафорические и т. д. \\
        \hline
        Доступность & свободно доступные, коммерческие, закрытые \\
        \hline
        Объем текстов & полнотекстовые, <<фрагментнотекстовые>> \\
        \hline
		\end{tabular}
        \end{adjustbox}
\end{table}

% Как устроены
\subsubsection{Устройство}

\subsection{Тексты}

% Виды текстовых разметок (?)
\subsubsection{Виды текстовых разметок}

% Структуру технических и проблемы в них
\subsubsection{Структура технических текстов и проблемы}

% Проблема терминов
\subsubsection{Проблема терминов}

Терминология русского языка складывалась и продолжает складываться на основе англоязычной терминологии.
Иногда получается, что давно существующий в русском языке термин приобретает новое значение, заимствованное из английского.
Так, например, слово <<корпус>> приобрело новое значение от английского <<corpus>> --- {сборник текстов} \cite{kl}.
В связи с этим, при разметке текстов, возникают трудности при сопоставлении терминов из одного языка с терминами другого.

Также, в связи с языковыми и культурными различиями, случается, что какой-то термин из одного языка не имеет точного эквивалента в другом.
Например, не имеет точного эквивалента в русском языке японский термин \begin{CJK*}{UTF8}{min}
侘寂
\end{CJK*}~(Wabi-Sabi), который означает {умение восхищаться чем-то несовершенным}.
Или наоборот, русский <<кипяток>> нельзя перевести на английский одним словом.

% Но даже если эквивалентный термин в другом языке существует, в силу различия выразительных способностей языков, может происходить потеря точности при переводе.
% Если в русскоязычном тексте встретится предложение <<зеленая зелень зеленит зеленую зелень>>, его можно будет просто перевести на английский, как <<green green green green green>>.
% С обратным же переводом возникнут проблемы.

\subsection{Разметки текстов}

% \subsection{Вывод}
