\phantomsection\section*{ОПРЕДЕЛЕНИЯ}\addcontentsline{toc}{section}{ОПРЕДЕЛЕНИЯ}

В настоящей расчетно-пояснительной записке применяют следующие термины с соответствующими определениями.

\uline{База данных} --- это некоторый набор перманентных (постоянно хранимых) данных, используемых прикладными программными системами какого-либо предприятия.~\cite{date}

\uline{Домен} --- это подмножество значений некоторого типа данных, имеющих определенный смысл.~\cite{lecnotes}

\uline{Атрибут отношения} --- это пара вида <имя\_атрибута, имя\_домена>.
Имена атрибутов должны быть уникальны в пределах отношения.~\cite{lecnotes}

\uline{Схема отношения} --- это именованное множество упорядоченных пар <имя\_атрибута, имя\_домена>.
Степенью или <<арностью>> схемы отношения является мощность этого множества.~\cite{lecnotes}

\uline{Кортеж, соответствующий данной схеме отношения} --- это множество упорядоченных пар <имя\_атрибута, значение\_атрибута>, которое содержит одно вхождение каждого имени атрибута, принадлежащего схеме отношения.
Значение атрибута должно быть допустимым значением домена, на котором определен данный атрибут.
Степень кортежа совпадает со степенью соответствующей схемы отношения.~\cite{lecnotes}

Отношение, определенное на множестве из N доменов содержит две части: заголовок (схему отношения) и тело (множество из M кортежей).
Значения N и M называются соответственно \uline{степенью} и \uline{кардинальностью} отношения.~\cite{lecnotes}

\uline{Потенциальный ключ} --- это непустое подмножество множества атрибутов схемы отношения, обладающее свойствами уникальности (в отношении нет двух различных кортежей с одинаковыми значениями потенциального ключа) и неизбыточности (никакое из собственных подмножеств множества потенциального ключа не обладает свойством уникальности).~\cite{lecnotes}

\uline{Первичный ключ} --- выбранный потенциальный ключ.~\cite{lecnotes}

\uline{Альтернативный ключ} --- потенциальный ключ, не являющийся первичным.~\cite{lecnotes}

\uline{Внешний ключ} в отношении R2 --- это непустое подмножество множества атрибутов FK этого отношения, такое, что:
\begin{enumerate}[label=\asbuk*)]
    \item существует отношение R1 с потенциальным ключом CK;
    \item каждое значение внешнего ключа FK в текущем значении отношения R2 обязательно совпадает со значением ключа CK некоторого кортежа в текущем значении отношения R1.~\cite{lecnotes}
\end{enumerate}

\uline{Реляционная база данных} --- это набор отношений, имена которых совпадают с именами схем отношений в схеме базы данных.~\cite{lecnotes}

\uline{Система управления базой данных} (СУБД) представляет собой программное обеспечение, которое управляет всем доступом к базе данных.~\cite{date}
